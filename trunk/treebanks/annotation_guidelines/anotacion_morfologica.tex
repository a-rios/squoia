\documentclass[a4paper,10pt]{scrartcl}
\usepackage[utf8x]{inputenc}
\usepackage[spanish]{babel}
\renewcommand\shorthandsspanish{}
\usepackage{covington}
\usepackage{graphics}
\usepackage{graphicx}
\usepackage{booktabs}
\usepackage{multirow}
\usepackage{varwidth}
\usepackage{arydshln}
\usepackage{lsalike}
\usepackage{hyperref}
\usepackage{tikz}
\usetikzlibrary{arrows,decorations.pathmorphing,backgrounds,fit,decorations.pathreplacing,fadings}
%opening
\title{Esquema de anotaciones morfol\'ogicas para el Quechua}
\author{Annette Rios}

\begin{document}

\maketitle

\begin{abstract}

\end{abstract}

\section{Anotaci\'on Morfolog\'ogica}

La anotaci\'on morfol\'ogica tiene dos niveles para cada parte de una palabra:\\
\begin{enumerate}
 \item un tipo de 'POS-Tag' (\emph{parts-of-speech}) que contiene la informaci\'on sobre la clase de morfema, v\'ease la tabla \ref{Ap:clases} en el ap\'endice \ref{Ap:morf}
  \item una etiqueta morfol\'ogica en el caso de sufijos o la clase de ra\'iz en el caso de las ra\'ices, v\'ease la tabla \ref{Ap:tags} el ap\'endice \ref{Ap:morf}
\end{enumerate}


La anotaci\'on sint\'actica se basa en unidades de grupos de morf\'emas, llamados \emph{inflectional groups}, en lo siguiente abreviado como IG.
En el esquema nominal de palabras tenemos 5 IG's, en el esquema verbal s\'olo 3, v\'ease tabla \ref{Tab:IG}.  
En el caso de los sufijos independientes o ambivalentes optamos por hacer de cada \emph{slot} un IG. Tenemos  7 \emph{slots} de sufijos independientes, ilustrado en la tabla \ref{Tab:Ambivalent Suffixes}.
Ejemplos:
\begin{examples}
 \item\label{Ex:qati} \gll qati -ra -mu -sha -qti -n -\~{n}a -s
      to.herd -Rptn -Dir -Prog -DS -3.Sg.Poss -Disc -IndE
      \glt'..when he was gone away herding (they say)..'
\glend
\item[6 IGs:] 
	\gll qatiramu -sha -qti -n -\~{n}a -s
	IG1 IG2 IG3 IG4 IG5 IG6
\glt
\glend

\item\label{Ex:wawa} \gll wawa -yki -kuna -wan
	   child -2.Sg.Poss -Pl -Inst
      \glt 'with your children'
\glend

\item[3 IGs:]
 \gll wawa -ykikuna -wan
      IG1 IG2 IG3
\glt	    
\glend

\end{examples}
\pagebreak
\begin{table}
\caption{Esquema nominal/verbal, IG's}\label{Tab:IG}
\begin{center}
        \begin{tabular}{ccp{3cm}ccc}

 & nominal scheme &  & verbal scheme & \\ \cmidrule[1pt]{1-2} \cmidrule[1pt]{4-5}
& \tikz[remember picture] \node (nw) {nominal root};& & \tikz[remember picture] \node (vw) {verbal root}; & \\
nom. IG 1 &  & & & verb. IG 1 \\
& derivation & & derivation &  \\ \cmidrule[1pt]{1-2} \cmidrule[1pt]{4-5}
\multirow{3}{1.7cm}{nom. IG 2}&poss. derivation& & object& verb. IG 2\\ \cmidrule[1pt]{4-5}
& plural& &  \tikz[remember picture] \node (a) {aspect}; & \multirow{4}{1.6cm}{verb. IG 3} \\ \cmidrule[1pt]{1-2} 
nom. IG 3&case 1& & tense &\\ \cmidrule[1pt]{1-2} 

nom. IG 4& case 2  &   & subject(object) &\\ \cmidrule[1pt]{1-2} 

nom IG 5 &\tikz[remember picture] \node (k) {case 3};   &  & \tikz[remember picture] \node (m) {mood};&\\ \cmidrule[1pt]{1-2} \cmidrule[1pt]{4-5} \addlinespace \addlinespace

        \end{tabular}
\end{center}
%\path[->] (start) edge node {c} (s1)
\begin{tikzpicture}[remember picture]
\path [overlay,bend right,->,very thick,blue,opacity=.3]
(a.west) edge node[below] {NS} (nw.south);
\path [overlay,bend right,->,very thick,red,opacity=.3]
(nw) edge node[above] {VS} (vw.south);
\end{tikzpicture}
\end{table}




\begin{table}
\begin{addmargin}{\dimexpr -\oddsidemargin-1in\relax}
    \caption{Sufijos ambivalentes/independentes, slots}\label{Tab:Ambivalent Suffixes}
   \vspace{0.4cm}
    \begin{center}
      \small
       \begin{tabular}{c:c:c:c:c:ll:ll}

\multicolumn{1}{c}{slot 1} & \multicolumn{1}{c}{slot 2}  &\multicolumn{1}{c}{slot 3}  & \multicolumn{1}{c}{slot 4} & \multicolumn{1}{c}{slot 5}  & \multicolumn{2}{c}{slot 6} & \multicolumn{2}{c}{slot 7}\\  \addlinespace
\hline
\multirow{3}{1cm}{\textit{hina} 'Sim'} & \multirow{6}{1cm}{\textit{puni} 'Cert'} & \multirow{2}{2cm}{\textit{pas/pis} 'Add'} &  \multirow{6}{1.7cm}{\textit{taq} \\'Con/Intr'} &  \multirow{6}{1.7cm}{\textit{chu} \\'Neg/Intr'} & \textit{mi} & 'DE'& \textit{iki} & 'Res'\\ \cline{6-9}
& &  \multirow{4}{2cm}{\textit{raq} 'Cont'} & & & \textit{si} & 'IE' & \multirow{2}{0.5cm}{\textit{\'A}} & \multirow{2}{0.7cm}{'Emp'}\\ \cline{6-7} \cdashline{3-3}
&  & & & & \textit{cha} & 'Ass' &\\ \cdashline{6-6} \cline{6-9}
\multirow{3}{1cm}{\textit{pacha} 'desde'}& & \multirow{4}{2cm}{\textit{\~{n}a} 'Disc'}  & & & \textit{qa} & \multicolumn{1}{l|}{'Top'}&  \multirow{3}{0.5cm}{\textit{ya}} & \multirow{3}{0.7cm}{'Emp'} \\ \cline{3-3} \cline{6-7}
& & & & & \textit{ri} & \multicolumn{1}{l|}{'QTop'} &  & \\ \cline{6-7}
& & & & & \textit{suna} & \multicolumn{1}{l|}{'Dub'}  \\ 
%\bottomrule \bottomrule
\hline 
        \end{tabular}
    \end{center}
   \vspace{0.2cm}
  \centering \scriptsize (dashed line - combination possible vs. normal line - combination not possible)\\
	
\end{addmargin}
\end{table}


\clearpage



\bibliographystyle{lsalike}
\bibliography{./quechua}

%\pagebreak
\appendix
 
\section{Etiquetas morfol\'ogicas}\label{Ap:morf}
\subsection{Clases de morfemas}\label{Ap:clases}

\begin{center}
\begin{tabular}{lll}
abreviaciones & ingl\'es & espa\~nol\\ \midrule
AS & ambivalent suffixes & sufijos ambivalentes (independientes\\
Asp & aspect & aspecto\\
NDeriv & nominal derivation suffixes & sufijos de derivaci\'on nominal\\
NPers & nominal person suffixes (possessives) &sufijos nominales de persona (posesivos\\
NS & nominalizing suffixes & sufijos nominalizadores\\
Num & number & n\'umero\\
Root & ra\'iz\\
Tns & Tense & tiempo \\
VDeriv & verbal derivational suffixes & sufijos de derivaci\'on verbal\\
VPers & verbal person suffixes & sufijos verbales de persona\\
VS & verbalizing suffixes & sufijos verbalizadores\\
\end{tabular}
\end{center}


\subsection{Morfolog\'ia}\label{Ap:tags}

\begin{center}
\begin{tabular}{llll}
abreviaciones & ingl\'es & espa\~nol\\ \midrule
\multicolumn{3}{c}{\underline{abreviaciones usadas en las etiquetas}}\\ \addlinespace
Excl & exclusive (in 1. person plural) & exclusiva (en 1. persona plural\\
Fut& future tense & futuro \\
Hab& habitual past & pasado habitual \\
Imp& imperative & imperativo \\
Incl& Inclusive & inclusiva (en 1. persona plural\\
IPst& past of indirect evidence & pasado de evidencialidad indirecta \\
Obj& object participant & objeto \\
Pl& plural & plural \\
Poss& possessive & posesivo \\
Pot& potential & potencial \\
NPst& evidentially) neutral past & pasado neutral \\
NRoot & nominal roots & ra\'ices nominales\\
NRootCMP & nominal compounds & compuestos nominales \\
NRootES & nominal root of Spanish origin & ra\'ices nominales espa\~noles\\
NRootNUM & nominal roots,numbers & ra\'ices nominales, n\'umeros\\
Part & particles & part\'iculas\\
PrnDem & demonstrative pronouns & pronombres demonstrativos \\
PrnInterr & interrogative pronouns & pronombres interrogativos\\
PrnPrs & personal pronouns & pronombres personales\\
Sg & singular & singular \\
Subj& subject participant & sujeto \\ 
VRoot & verbal roots & ra\'ices verbales\\
VRootES & verbal roots of Spanish origin & ra\'ices verbales espa\~noles\\
\end{tabular}

\begin{tabular}{llll}
\multicolumn{3}{c}{\underline{etiquetas de sufijos/part\'iculas individuales}}\\ \addlinespace

+Abl & ablative case ('from') & caso ablativo & -manta\\
+Abss & abessive ('without') & abesivo &-nnaq\\
+Acc & accusative case  & acusativo & -ta\\
+Add & additive ('too, even') & aditivo  & -pas\\
+Aff & affective & afectivo &  -yku\\
+Affir & affirmative & affirmativo & arí \\
+Ag & agentive ('nomen agentis') & agentivo & -q\\
+Aprx & approximative ('near, about')& approximativo & -niq\\
+Asmp & assumption & suposici\'on,asunci\'on & -ch/cha\\
+Asmp{\textunderscore}Emph & empathic assumption & suposici\'on,asunci\'on emp\'atica & -chá\\
+Ass & assistive & asistivo & -ysi/-schi\\
+Aug & augmentative & augmentativo & -su \\ 
+Autotrs & autotransformative & autotransformativo & -lli\\
+Ben & benefactive & benefactivo & -paq\\
+Caus & causative & causativo & -chi\\
+Char & characterization & caracterizaci\'on & -li/-raya\\
+Cis{\textunderscore}Trs & cislocative/translocative & cislocativo/translocativo & -mu\\
+Con{\textunderscore}Inst & connective/instrumental & conectivo/instrumental &  -wan\\
+Con{\textunderscore}Intr & contrastive/interrogative & contrastivo/interrogativo& -taq\\
+Cond & conditional particle 'si' (Spanish loan)\\
+Conec & conective particle & part\'icula conectiva & ima\\
+Cont & continuative  & continuativo & -nya\\
+Contr & contrastive particle ('or') & part\'icula contrastiva &icha\\
+DS & different subject  & sujeto diferente & -qti/pti \\
+Dat{\textunderscore}Ill & dative/illative & dativo/illativo & -man\\
+Def & definitively & defintivo & -puni\\
+Des & desiderative & desiderativo &    -naya\\
+Desesp & desesperative & desesperativo &  -pasa\\

\end{tabular}


\begin{tabular}{llll}
+Dim & diminutive & diminutivo & -cha\\
+DirE & direct evidence & evidencialidad directa & -mi/n\\
+DirE{\textunderscore}Emph & empathic direct evidence & evidencialidad directa emp\'atica & -má\\
+Disc & discontinuative  & discontinuativo & -ña\\
+Dist & distributive, nominal suffix & distributivo (nominal)  & -kama\\
+Distr & distributive, verbal suffix  & distributivo (verbal) & -nka\\
+Dub & dubitative &dubitativo  & -suna\\
+Emph & empathic  & emp\'atico & -yá\\
+Fact & factitive &factitivo & -cha\\
+Fut & future & futuro \\
+Gen & genitive case & caso genitivo & -pa\\
+Hon{\textunderscore}Aff & honorific/affective/limitative & honor\'ifico/afectivo/limitativo & -lla\\
+IPst & past of indirect evidence & pasado de evidencialidad indirecta  & -sqa\\
+Iclsv & inclusive case & caso inclusivo & -ntin\\
+Imp & imperative & imperativo \\
+Inch & inchoative & incoativo & -ri\\
+IndE & indirect evidence & evidencialidad indirecta  & -si/s\\
+IndE{\textunderscore}Emph & empathic indirect evidence & evidencialidad indirecta emp\'atica & -sá\\
+Inf & infinitive & infinitivo & -y\\
+Int & intentional & intencional & -rpari\\
+Intr & interrogative & interrogativo & -chu\\
+Intr{\textunderscore}Neg & interrogative/negation & interrogativo/negaci\'on & -chu\\
+Intrup & interruptive & interruptivo & -ykacha/kacha\\
+Intsoc & intersociative & intersociativo & -pura\\
+Kaus & cause & causa & -rayku\\
+Loc & locative & locativo & -pi\\
+MPoss & multi-possessor & multi-posesor & -sapa\\
+MRep & multi-repetitive & multi-repetitivo & -paya\\
+NPst & neutral past & pasado neutral & -rqa/ra\\
+Neg & negation & negaci\'on & -chu, mana, ama\\
+Neg{\textunderscore}Emph & empathic negation particle & negaci\'on emp\'atica & maná\\
+Neg{\textunderscore}Imp & imperative negation particle & negaci\'on imperativa &ama\\
+NumOrd & ordinal numeral marker & n\'umero ordinal &ñiqin\\
+Obl & obligation & obligaci\'on  & -na\\
+Perdur & perdurative & perdurativo & -raya\\
+Perf & perfect & perfecto & -sqa\\
+Pl & plural & plural & -kuna\\
+Posi & positional & posicional & -mpa\\
+Poss & possessive & posesivo \\
+Pot & potential & potencial & -man\\
+Prog & progressive & progresivo & -sha\\
+Proloc & prolocative & prolocativo  & -nta\\
+QTop & topic marker in questions & t\'opico en preguntas & -ri\\
+Rel & relational & relacional & -n\\
+Rem & rememorative & rememorativo & -ymana\\
+Rep & repetitive & repetitivo & -pa\\
+Res & resignation particle/suffix & resignaci\'on & -iki\\
\end{tabular}

%\pagebreak
\begin{tabular}{llll}
+Reub & reubicative & reubicativo & -na\\
+Rflx{\textunderscore}Int & reflexive/intensifier  & reflexivo/intensificador & -ku\\
+Rgr{\textunderscore}Iprs & regressive/interpersonal & regresivo/interpersonal & -pu\\
+Rptn & repentine ('surprise, haste, urgency') &repentino & -rqu/ru\\
+Rzpr & reciprocal & rec\'iproco & -na\\
+SS & same subject & sujeto id\'entico & -spa\\
+SS{\textunderscore}Sim & simultaneous same subject & sujeto id\'entico y acci\'on simult\'anea & -stin\\
+Sim & similarity & similaridad & -hina, -niraq, hina\\
+Sim{\textunderscore}Disk & simulative/discontinuative & simulativo/discontinuativo & -tiya\\
+Sml & simulative & simulativo  & -ykacha/kacha\\
+Soc & sociative & sociativo & -puwan\\
+Term & terminative & terminativo & -kama\\
+Top & topic marker & t\'opico & -qa\\
+Trs & transformative & transformativo & -ya\\
+Vdim & verbal diminutive ('childishness') & diminutivo verbal & -cha\\
+Abtmp & abtemporal particle ('since')& part\'icula abterminal & pacha\\

\end{tabular}
\end{center}



\end{document}
